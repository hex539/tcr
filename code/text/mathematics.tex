\section {General theorems}

  \subsection {Erd\H{o}s-Gallai theorem (validity of a degree sequence)}
    A sequence of non-negative integers $ d_1 \geq \ldots \geq d_n $ can be interpreted as the
    degree sequence of a finite simple graph on n vertices \emph{if and only if}
    $ \big( d_1+\cdots+d_n \big) $
    is even and
    $ \big( \sum^{k}_{i=1}d_i\leq k(k-1)+ \sum^n_{i=k+1} \min(d_i,k) \big) $
    holds for all
    $ 1\leq k\leq n $.

  \subsection {Kirchhoff's theorem (counting spanning trees)}
    Let $G$ be a connected graph with $n$ labelled vertices, and $\lambda_1$, $\lambda_2$, ..., $\lambda_{n−1}$ be the non-zero eigenvalues of its Laplacian;
    then the number of spenning trees of $G$ is equal to $ \big( t(G)=\frac{1}{n} \cdot \lambda_1 \cdot \lambda_2 \cdot \ldots \cdot \lambda_{n-1} \big) $.

    More usefully, this count is equivalent to the determinant of any cofactor (ie. a truncation of the matrix to $ (n-1) \cdot (n-1) $) - which is solved in
    $ O(N^3) $ time via Gaussian elimination.

    The Laplacian (or Kirchhoff) matrix can be calculated as $ \big( L_{ij} = D_{ij} - A_{ij} \big) $, where $D$ is the degree matrix and $A$ is the adjacency matrix.

  \subsection {K\"{o}nig's theorem (minimum path cover in a DAG)}
    The number of non-intersecting paths needed to completely cover the vertices of a DAG (Directed Acyclic Graph) is equal to the number of vertices in the DAG
    minus the cardinality of a maximum matching in its corresponding bipartite graph.

    A vertex can either be the terminal node in a path, or part of exactly one path belonging to a successor. The trick is to equate the number of succeeded nodes
    with a maximum matching of these "source" vertices with corresponding "sink" vertices (each vertex is split into both types, one in each half of a bipartite
    graph).

    The minimum number of paths is then equal to the number of terminal (unmatched) nodes.

    Note that Dilworth's theorem regarding the longest antichain in a partially ordered set (the biggest subset of incomparable elements in a DAG, dependency graph
    or partial ordering) is equivalent to this; the result is calculated by relaxing all edges then returning the size of the minimum path cover.

  \subsection {Sprague-Grundy theorem (impartial gameplay)}
    Every piece in the state of an impartial game has an associated Grundy number, which is theoretically equivalent to a Nim heap; Nim can be won in perfect play
    starting from a state \emph{if and only if} it is not of the form $ \big( s_{1} \oplus s_{2} \oplus \ldots \oplus s_{n-1} \oplus s_{n} = 0 \big) $, by moving to such
    a state, where this will never be the case.

    The \emph{Grundy number} for individual components of the game state (e.g.\ moving a pawn in chess) is equal to the smallest non-negative integer that is not
    associated with a state that can be reached within one move. This is straightforward to find by checking each candidate number in turn against the set of moves.

\section {General observations}

  \subsection {Cactus graphs}
    A graph in which any two simple cycles have at most one vertex in common is called a \emph{cactus graph}.

    It's possible to identify the cycles in these graphs by running a depth-first search and keeping track of the depth for each vertex. When a newly-processed edge
    from a vertex with depth $A$ is incident on an already-processed vertex with depth $B$, that vertex and its $ (A - B) $ parents in the callstack all belong to the
    same cycle.

    It's often easier to solve query problems on cactus graphs by contracting all cycles into 'super-vertices'.

  \subsection {Two-satisfiability}
    2SAT is a special case of SAT where all clauses are of the form
    $(a \lor \neg b) \land (\neg a \lor c) \land \dots \land (d \lor e)$,
    and can be solved by making a graph where each boolean variable is represented by two vertices:
    one for position, one for negation; for statements such as $(a \lor b)$, make an implication
    (directed edge) from $[\neg a \rightarrow b]$ and from $[a \rightarrow \neg b]$ (this is called
    an implication graph. I wonder why).

    When we take the SCC graph, the set of clauses is solvable if and only the same variable
    doesn't appear twice in the same SCC (ie. root$[2 \cdot x]$ != root$[2 \cdot x + 1] \forall x$).
    We can then find a satisfying solution by traversing the dependency graph, and whenever we find an
    SCC that doesn't have its variables set yet (all must be the same, obviously) we'll set all of
    them to FALSE and set the SCCs of converses (ie. $\neg a \rightarrow a$) to TRUE. If an SCC is entirely
    TRUE then so are all SCCs reachable from it.

  \subsection {Tree canonicalisation}
    \subsubsection {Rooted}
      A rooted tree is uniquely identified as the multiset of its subtrees. By assigning indices to ordered multisets as they are processed, subroots can be
      labelled bottom-up in $ O(N \cdot \log{N}) $ time (there exist linear-time algorithms; as with suffix arrays these are overkill).

    \subsubsection {Unrooted}
      The central vertices of an unrooted tree are those with minimum distance to any other vertex; there can be at most two of these and they can be identified by
      using breadth-first dependency traversal to eliminate outer layers repeatedly until fewer than 3 vertices remain - now the label for this tree is a tuple of the
      remaining set labels when the tree is rooted at each in turn.

\section {Useful formulae}

  \subsection {Power sums}
    $ {\sum^n_{x=1} x^1} = (n \cdot n + n) \div 2 $

    $ {\sum^n_{x=1} x^2} = (2 \cdot n^3 + 3 \cdot n^2 + n) \div 6 $

    $ {\sum^n_{x=1} x^3} = (n^4 + 2 \cdot n^3 + n^2) \div 4 $

    $ {\sum^n_{x=1} x^4} = (6 \cdot n^5 + 15 \cdot n^4 + 10 \cdot n^3 - n) \div 30 $

  \subsection {Determinants}
    $ \left| \begin{array}{cc} a & b \\ c & d \\ \end{array} \right| = a \cdot d - b \cdot c $

    $ \left| \begin{array}{ccc} a & b & c \\ d & e & f \\ g & h & i \\ \end{array} \right| = a \cdot (e \cdot i - f \cdot h) + b \cdot (f \cdot g - i \cdot d) + c \cdot (d \cdot h - e \cdot g) $

\section {Things that are approximately 10,000,000}

  $ (10)! $;
  $ {26} \choose {13} $;
    fib($ 34 $);
    cat($ 14 $) = unique 15-node binary trees $ \Big[ C_n = {2n \choose n} - {2n\choose n+1} \quad | n \ge 0 \Big] $

  $ (\sqrt{2})^{46} $;
  $ (2)^{23} $;
  $ (\sqrt{3})^{29} $;
  $ (3)^{14} $

  $ (10,000)^{1.75} $;
  $ (3,000)^{2} $;
  $ (250)^{3} $;
  $ (100)^{3.5} $;
  $ (50)^{4} $;
  $ (30)^{4.5} $;
  $ (15)^{6} $

\section {Primes}
  $251\,429$

  $11\,111\,117$

  $16\,431\,563$

  $2\,147\,483\,647$

  $3\,111\,111\,111\,113$

  $6\,660\,000\,000\,001$

  $2\,345\,678\,911\,111\,111$

  $9\,999\,999\,900\,000\,001$
